%Register Tabelle
\subsection{Memory APB3 Registers}
\subsubsection{Register Mapping}
\label{sec:apbRegDec}
\renewcommand{\arraystretch}{1.5}
\begin{longtable} [htb] { |p{0.3\linewidth}|p{0.14\linewidth}|p{0.14\linewidth}|p{0.3\linewidth}| } \hline
	\label{tab:table3}
	\textbf{Name} & \textbf{Addr}& \textbf{R/W}&\textbf{Describtion}\\
	\hline
	Stamp1ShadowReg1&0x04&R + W& Reset on write. \par Reset to 0x0.\\
	\hline
	Stamp1ShadowReg2&0x08&R + W& Reset on write. \par Reset to 0x0.\\
	\hline
	Stamp2ShadowReg1&0x0C&R + W& Reset on write. \par Reset to 0x0.\\
	\hline
	Stamp2ShadowReg2&0x10&R + W& Reset on write. \par Reset to 0x0.\\
	\hline
	Stamp3ShadowReg1&0x14&R + W& Reset on write. \par Reset to 0x0.\\
	\hline
	Stamp3ShadowReg2&0x18&R + W& Reset on write. \par Reset to 0x0.\\
	\hline
	Stamp4ShadowReg1&0x1C&R + W& Reset on write. \par Reset to 0x0.\\
	\hline
	Stamp4ShadowReg2&0x20&R + W& Reset on write. \par Reset to 0x0.\\
	\hline
	Stamp5ShadowReg1&0x24&R + W& Reset on write. \par Reset to 0x0.\\
	\hline
	Stamp5ShadowReg2&0x28&R + W& Reset on write. \par Reset to 0x0.\\
	\hline
	Stamp6ShadowReg1&0x2C&R + W& Reset on write. \par Reset to 0x0. \\
	\hline
	Stamp6ShadowReg2&0x30&R + W& Reset on write. \par Reset to 0x0.\\
	\hline
	SyncStatusReg&0x34&R + W& Look at section \ref{sec:syncStatusReg}.\\
	\hline
	ConfigReg&0x38&R + W& Look at section \ref{sec:configReg}. \\
	\hline
	ResetTimerValueReg&0x3C&R + W& 32 bit value. \par Default at 0xFFFFFFFF. \par Counts clock rising edges. \\
	\hline
	WaitingTimerValueReg&0x40&R + W&  32 bit value. \par Default at 0xFFFFFFFF. \par Counts clock rising edges. \\
	\hline
	ResyncTimerValueReg&0x44&R + W&  32 bit value. \par Default at 0xFFFFFFFF. \par Counts clock rising edges.\\
	\hline
	TimestampReg&0x48&R + W& Reset on write. \par Reset to 0x0.\\
	\hline
	SyncStatusReg2&0x4C&R + W& Reset on write. \par Reset to 0x0. \par Look at section \ref{sec:SyncStatusReg2}.\\
	\hline
\caption[Register map]{Register Map}
\label{tbl:registeraddress}
\end{longtable}
\renewcommand{\arraystretch}{1.0}
\subsection{SyncStatus Register} 
\label{sec:syncStatusReg}
\begin{figure}[htb]
	\begin{center}
		\begin{tikzpicture}
			\bitrect{16}{16 - \bit}
			\rwbits{0}{4}{RE}
			\rwbits{4}{1}{R6}
			\rwbits{5}{1}{R5}
			\rwbits{6}{1}{R4}
			\rwbits{7}{1}{R3}
			\rwbits{8}{1}{R2}
			\rwbits{9}{1}{R1}
			\rwbits{10}{1}{M6}
			\rwbits{11}{1}{M5}
			\rwbits{12}{1}{M4}
			\rwbits{13}{1}{M3}
			\rwbits{14}{1}{M2}
			\rwbits{15}{1}{M1}
			
		\end{tikzpicture}
	\end{center}
	\caption[SyncStatus Register bits 0 to 15]{SyncStatus Register bits 0 to 15}
	\label{fig:SSR11}
\end{figure}
\begin{figure}[htb]
	\begin{center}
		\begin{tikzpicture}
			\bitrect{16}{32 - \bit}
			\rwbits{0}{1}{PS}
			\rwbits{1}{1}{PR}
			\robits{2}{1}{}
			\rwbits{3}{1}{AE}
			\robits{4}{2}{}
			\rwbits{6}{1}{O6}
			\rwbits{7}{1}{O5}
			\rwbits{8}{1}{O4}
			\rwbits{9}{1}{O3}
			\rwbits{10}{1}{O2}
			\rwbits{11}{1}{O1}
			\rwbits{12}{4}{RE}			
		\end{tikzpicture}
	\end{center}
	\caption[SyncStatus Register bits 31 to 16]{SyncStatus Register bits 31 to 15}
	\label{fig:SSR12}
\end{figure}
\noindent
M1 - M6: Bitmask for every stamp which has not provied a newAvails signal. \\
R1 - R6: Bitmask for every stamp which is requesting a Resync. \\ 
RE: 8 bit counter: the number of ResyncEvents. \\ 
O1 - O6: StatusReg2 overflow marker. Means that the difference to the timestamp register is bigger than the size of 5 bits. \\ 
AE: APB Error Address not known. \\ 
PR: Pending Reading Interrupt. \\
PS: Pending Synchronizer Interrupt. \\
\subsection{SyncStatus Register 2} 
\label{sec:SyncStatusReg2}
\begin{figure}[htb]
	\begin{center}
		\begin{tikzpicture}
			\bitrect{16}{16 - \bit}
			\rwbits{0}{1}{S4}
			\rwbits{1}{5}{S3}
			\rwbits{6}{5}{S2}
			\rwbits{11}{5}{S1}			
		\end{tikzpicture}
	\end{center}
	\caption[SyncStatus Register 2 bits 15 to 0]{SyncStatus Register 2 bits 15 to 0}
	\label{fig:SSR21}
\end{figure}
\begin{figure}[htb]
	\begin{center}
		\begin{tikzpicture}
			\bitrect{16}{32 - \bit}
			\robits{0}{2}{}
			\rwbits{2}{5}{S6}
			\rwbits{7}{5}{S5}	
			\rwbits{12}{4}{S4}			
		\end{tikzpicture}
	\end{center}
	\caption[SyncStatus Register bits 31 to 16]{SyncStatus Register bits 31 to 15}
	\label{fig:SSR22}
\end{figure}
S1 - S6: 5 bit counter; relative distance to the Timestamp value. \\
\subsection{Config  Register} 
\label{sec:configReg}
\begin{figure}[htb]
	\begin{center}
		\begin{tikzpicture}
			\bitrect{16}{16 - \bit}
			\robits{0}{9}{}
			\rwbits{9}{3}{RS value}
			\robits{12}{1}{}
			\rwbits{13}{3}{NA value}			
		\end{tikzpicture}
	\end{center}
	\caption[Config Register bits 15 to 0]{Config Register bits 15 to 0}
	\label{fig:CR1}
\end{figure}
\begin{figure}[!htb]
	\begin{center}
		\begin{tikzpicture}
			\bitrect{16}{32 - \bit}
			\rwbits{0}{1}{EI}
			\rwbits{1}{1}{CS}
			\rwbits{2}{1}{CR}	
			\robits{3}{13}{}			
		\end{tikzpicture}
	\end{center}
	\caption[Config Register bits 16 to 31]{Config Register bits 31 to 16}
	\label{fig:CR2}
\end{figure}
\noindent
NA value: Number of minimal neccessary \textit{newAvail} signals. Default 4. \\
RS value: Number of minimal neccessary \textit{Request Resync} signals. Default 4. \\ 
CR: Clear Read Interrupt. \\
CS: Clear Synchronizer Interrupt. \\ 
EI: Enable Interrupts. \\
\subsection{Calculating Timer Values}
\label{sec:CTV}
This only affects the registers \textit{ResetTimerValue}, \textit{WaitingTimerValue} and \textit{ResyncTimerValue}. \\ 
The timer are designed as a countdown timer, if they are reaching the value zero, their action will be executed. The resolution of the timers is 20 ns, so the values will be calculated like equation \ref{eq:timer}.
\begin{equation}
	\begin{split}
			\text{Value} = \frac{\text{Excpected Time}}{T_{cpu}} = \frac{\text{Excpected Time}}{20 ~ns} 
	\end{split}
\label{eq:timer}
\end{equation}	