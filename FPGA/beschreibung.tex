\subsection{Komponente: Stamp}
\label{bes:Stamp}
Die STAMP Systemkomponente beschreibt einen Logikblock, der für die Kommunikation mit dem ADS 1147 zuständig ist. Siehe Abbildung \ref{fig:Stamp}. 
Zur Kommunikation mit dem ADS 1147 Analog - Digital - Wandlern (ADC) , wird ein SPI Bus genutzt. Der ADC wird durch die Komponente Stamp eingestellt, die entsprechenden Werte können durch den angebrachten AMBA Bus geladen werden. Die entsprechenden Werte werden von dem im SoC integrierten Microcontroller geladen werden. 
Der die beiden ADCs, welche mit dem dem Dehnungsmessstreifen (SGR) verbunden sind werden mit einer Geschwindigkeit von 2 kSp/s gewandelt. Der ADC an dem PT100 wird mit 10 Sp/s betrieben. 
% TODO: Mehr dazu schreiben, Jonathans komponente
\subsection{Komponente: Memory}
\label{bes:memory}
\subsection{Komponente: Watchdog}
Der Watchdog implementiert seinen eigenen Timer zur Überwachung und Steuerung der Stamp - Komponente. Der Watchdog Timer wird auf 2 ms eingestellt. In dieser Zeit sollten die Stamps vier Werte geliefert haben. Nachfolgendes Diagramm zeigt das zeitliche Verhalten des Systems. 
\renewcommand{\arraystretch}{1.5}
\begin{table}[htb]
  \begin{tabular}{|p{.3\linewidth}|p{.15\linewidth}|p{.45\linewidth}|}\hline
  \textbf{Signal} & \textbf{Richtung} & \textbf{Beschreibung} \\ \hline
  Async. Reset n & output & Watchdog leitet $\overline{reset}$ an die Komponenten weiter. Siehe Abbildung XXX. \\ \hline 
  AMBA & input / output & Busanbindung an den Microcontroller zur Einstellung des Watchdogs und zur Übermittlung von Informationen über eine Komponente, z.B. nach Async. Reset. \\ \hline
  Data Ready n & input & Setzt den Watchdog für die entsprechende Komponente zurück. Erfolgt vier mal hintereinander kein Data Ready Signal wird die Komponente zurück gesetzt. Ein Identifier wird an den Microcontroller gemeldet, dieser hat nun die Konfiguration erneut zu laden. \\ \hline
  clk & input & Die Systemclock. \\ \hline 
  $\overline{reset} $ & input & Async. Reset des Gesamtsystems, der Reset führt auch zu einem Reset des Watchdogs. In Folge dessen wird auch der Async. Reset n für die einzelnen Stamp  - Komponenten getriggert. \\ \hline
  \end{tabular}
   \caption{Watchdog Signal Beschreibungen}
\end{table}
\renewcommand{\arraystretch}{1.0}

\begin{figure}[ht] 
\begin{center}
\begin{tikzpicture}[node distance = 4cm, black, ultra thick]
\node[state] (Init) {Init};
\node[state, above right of = Init] (S1) {S1}; %node[below of = S1, node distance = .5cm] {Name} ;
%\path ($(S1) + (-0.7, 0.5)$) --  ($(S1) + (0.7, 0.5)$);
\node[state, below right of = S1] (S2) {S2};
\node[state, below left  of = S2] (S3) {S3};
\coordinate (leftInit) at ($(Init) + (-2.5cm, 1cm) $);
\path[->]  (leftInit) edge node[sloped, anchor=center, above] {\small{$\overline{reset}$}} (Init);
\path[->] (Init) edge[bend left] node[sloped, anchor=center, above] {Timer$_{Init}$ Hit} (S1);
\path[->] (S1) edge[bend left] node[sloped, anchor=center, above] {pending = 1} (S2);
\path[->] (S2) edge[bend left] node[sloped, anchor=center, below] {Wdt hit} (S3);
\path[->] (S3) edge[bend left] node[sloped, anchor=center, above] {DRDY = 1} (S2);
\path[->] (S3) edge[bend left] node[sloped, anchor=center, below] {Counter = 4} (Init);
%
\path[->] (S1) edge[loop above] node[sloped, anchor=center, above] {pending = 0} (S1); 
\path[->] (S2) edge[loop right] node[sloped, anchor=center, above] {DRDY = 1} (S2); 
\path[->] (S3) edge[loop below] node[sloped, anchor=center, below] {Wdt hit} (S3); 
\path[->] (Init) edge[loop left] node[anchor=center, below left ] {Timer$_{Init} pending$} (Init); 
\end{tikzpicture}
\caption{Statemachine für ein Data Ready (DRDY) Signal}
\label{fig:wdtStatemachine}
\end{center}
\end{figure}
\renewcommand{\arraystretch}{1.5}
\begin{table}[htb]
  \begin{tabular}{|p{.5\linewidth}|p{.5\linewidth}|}\hline
 	 \textbf{State}  &\textbf{Aufgabe}  \\ \hline
 	Init	& $async.~ Reset = 1$ \par Timer$_{Init}$ start\\ \hline
	S1 &  $async.~ Reset = 0$ \par Timer$_{Init}$ stop \par AMBA Kommunikation zum $\mu$C: Initialisierung des entsprechenden Stamps. \\ \hline
	S2 & Watchdog Timer start \\ \hline
	S3 & Zähler hoch zählen\\ \hline
  \end{tabular}
   \caption{Watchdog Aufgaben der Zustände}
\end{table}
\begin{table}[htb]
  	\begin{tabular}{| p{.21\linewidth} | 
				 p{.04\linewidth} | 
				 p{.04\linewidth} | 
				 p{.04\linewidth} | 
				 p{.04\linewidth} | 
				 p{.04\linewidth} | 
				 p{.04\linewidth} | 
				 p{.04\linewidth} | 
				 p{.04\linewidth} | 
				 p{.04\linewidth} | 
				 p{.04\linewidth}|}\hline 
	% end of Description
		\textbf{Ausgangszustand} & 
		\multicolumn{2}{c|}{\textbf{$\overline{reset}$}} &  
		\multicolumn{2}{c|}{\textbf{Timer$_{Init}$ Hit}} &  
		\multicolumn{2}{c|}{\textbf{Wdt Hit}} &
		\multicolumn{2}{c|}{\textbf{ $DRDY_n$}} & 
		\multicolumn{2}{c|}{\textbf{AMBA Pending}}\\ \hline 
		% New line
		& 0 & 1 & 0 & 1 & 0 & 1 & 0 & 1 & 0 & 1\\ \hline
		% 
		Nicht definiert & Init & & & & & & & & &  \\ \hline
		Init &  & & Init & S1 & & & & & & \\ \hline
		S1 &  & & &  & & & &  & S1 & S2 \\ \hline
		S2 &  & & &  & & & S3 & S2 &  &  \\ \hline
		S3 &  & & &  & & S3 | Init &  & S2 &  &  \\ \hline
	\end{tabular}
   \caption{Watchdog Übergangstabelle}
\end{table}
\renewcommand{\arraystretch}{1.0} 
\noindent
Der Watchdog wird durch eine Statemachine für jede Stampkomponente definiert. Mit dem Signal $\overline{reset}$ wird der Initialzustand \textit{Init} erreicht. Im Zustand \textit{Init} wird der Timer$_{Init}$ gestartet. Dieser läuft 100 ns, danach wird das Signal $Timer_{Init}Hit = 1$ gesetzt und die Statemachine wechselt in den Zustand \textit{S1}. Der Zustand \textit{S1} beendet den Timer$_{Init}$ und erwirkt eine Initialisierung der entsprechende Stampkomponente. Mit dem Signal \textit{Config compl.}  (AMBA Pending = 1) wechselt der Zustand zu \textit{S2}. Im Zustand \textit{S2} wird der Watchdogtimer (Wdt) aktiviert. Erfolgt vor Ablauf des Timers das Signal \textit{DRDYn = 1} verbleibt die Statemachine Zustand \textit{S2} und der Watchdogtimer wird neugestartet. Erfolgt das Signal \textit{DRDY = 1} nicht, wechselt der Zustand auf \textit{S3}. 
Im Zustand \textit{S3} wird ein Zähler um eins inkrementiert. Mit dem Signal  \textit{DRDY = 1} wechselt der Zustand zurück auf \textit{S2} und der Zähler wird Null gesetzt. Läuft der Watchdogtimer erneut ab, so verbleibt der Zustand \textit{S3} und der Zähler wird weiter hochgezählt. Erreicht der Zähler den Wert 4 wird der Zustand auf \textit{Init} geändert. 

