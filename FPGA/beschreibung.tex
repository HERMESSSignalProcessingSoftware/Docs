\section{Component: Stamp}
\label{bes:Stamp}
The \textit{component STAMP} describes the logic used for the communication with the ADS 1147. (Look at figure \ref{fig:stamp}).  
SPI is used for the communication with the ADS 1147 analog - digial converter (ADC). The configuration of the ADC will be done from the microcontroller subsystem (MSS), therefore it will write to the \textit{STAMP} via APB3. \\ \\
During the measurment operations, this component will trigger an synchron signal \textit{New Avail}, signaling that the values of the ADCs are loaded successfully into the \textit{STAMP} component. On a rising edge of \textit{new avail} the data will be shown in the 64 bit databus. \\\\
The strain gauge rosetts (SGR) will be sampled with 2000 SP / s. The temperature sensor will be sampled with 10 SP/s. 
% TODO: Mehr dazu schreiben, Jonathans komponente
\section{Komponente: Memory}
\label{bes:memory}
\subsection{Controll Unit}
The \textit{Controll Unit} (CU) implements the syncrhonizer finit statemachine to resolve smaller problems during measurment operations. The CU will be initialized with a falling edge on $\overline{reset}$. After that it is able to do parallel APB3 and measurment to memory operations. \\ \\
The data will be stored on a 1Gbit external flash memory, connected to the MSS. 
\begin{figure}[htb] 
	\begin{center}
		\begin{tikzpicture}[node distance = 4cm, black, ultra thick]
			\node[state] (start) at (0,0) {Start};
			\node[state, above right of = start, node distance = 8cm] (ResyncEvent) {Resync Event};
			\node[state, below of = ResyncEvent, node distance = 8cm] (S11) {S11};
			\node[state, below left of = S11] (S1) {S1};
			\node[red!50, state, above left of = start] (Error1) {Error 1};
			\node[state, below of = start, node distance = 8cm] (End1) {End 1};
			% Draw lines 
			\coordinate (tempStart) at ($(start) - (2,-1)$);
			\path[->, draw, black] (tempStart) -- node[sloped, anchor=center, above, align=left]{$\overline{reset}$} (start);
			\path[->] (start) edge[loop left]  node[sloped, anchor=center, above,text width=3.5cm, align=left] {all other}(start);
			\path[->] (start) edge[bend left] node[sloped, anchor=center, above,text width=3.5cm, align=left]{\small{\# Request Resync = $X_1$\\ AND \\ ResyncTimer = 0}} (ResyncEvent);
			\path[->] (ResyncEvent) edge[bend left] (start);
			\path[->] (start) edge[bend left] node[sloped, anchor=center, above,text width=3.2cm, align=left] {\small{\# new Avail $>$ 0}} (S1);
			\path[->] (S1) edge[loop left] node[sloped, anchor=center, above, align = left] {\small{all other}} (S1);
			\path[->] (S1) edge[bend left] node[sloped, anchor=center, below, text width=3.2cm,align = left] {\small{\# New Avail = 6 \\ OR \\ $~$\\ \# New Avail $\ge~X_2$ \\ AND \\ WaitingTimer = 0 }} (End1);
			\path[->] (S1) edge[bend right] node[sloped, anchor=center, below, text width=3.2cm,align = left] {\small{WaitingTimer = 0 \\ AND \\ \# New Avail < $X_2$}}(S11);
			\path[->] (S11) edge[bend right] node[sloped, anchor=center, below, text width=3.2cm,align = left] {ResyncTimer = 0} (ResyncEvent);
			\path[->] (S11) edge[bend right] node[sloped, anchor=left, above, text width=3.2cm,align = left] {\small{ResyncTimer $\neq$ 0}}(start);
			\path[->] (start) edge[bend right] node[sloped, anchor=center, above, text width=3.2cm,align = left]{\small{SyncTimer = 0}} (Error1);
			\path[->] (End1) edge[bend left] (start);
		\end{tikzpicture}
	\end{center}
	\label{fig:MemCu}
	\caption[Memory controll unit statemachine]{Memory controll unit finit statemachine}
\end{figure}
\begin{landscape}
	\renewcommand{\arraystretch}{1.5}
	\setlength\LTcapwidth{\textwidth} % default: 4in (rather less than \textwidth...)
	\setlength\LTleft{0pt}            % default: \parindent
	\setlength\LTright{0pt}           % default: \fill
	\begin{longtable}[ht]{@{\extracolsep{0pt}}|l||*{14}{l|}}\hline 
		\textbf{Status} & 
		\multicolumn{2}{c|}{\textbf{$\overline{reset}$}} & 
		\multicolumn{4}{c|}{\textbf{\# Request Resync} \&\& \textbf{ ResyncTimer} } & \multicolumn{2}{c|}{\textbf{\# New Avail}} & 
		\multicolumn{4}{c|}{\textbf{WaitingTimer} \& \& \textbf{\# New Avail}} & 
		\multicolumn{2}{c|}{\textbf{SyncTimer} } \\ \hline 
		&
		\multicolumn{2}{c|}{} & 
		\multicolumn{2}{c|}{\textbf{\# Request Resync}} & \multicolumn{2}{c|}{\textbf{ ResyncTimer*} } & \multicolumn{2}{c|}{} & 
		\multicolumn{2}{c|}{\textbf{WaitingTimer}} & \multicolumn{2}{c|}{\textbf{\# New Avail}} & 
		\multicolumn{2}{c|}{} \\ \hline 
		\textbf{Wert} & 
		0 & 1 & 
		== $X_1$ & $\neq X_1$ & 
		== 0 & $\neq 0$ &
		> 0 & == 6 &
		== 0 & $\neq 0$ &
		> 0 \&\&  $\geq ~X_2$ & > 0 \&\& < $X_2$ & 
		\multicolumn{2}{c|}{ 0} \\ \hline 
		\endhead
		Undefined   & 
		Start &  &  & & &  & &  & &  & & & 
		\multicolumn{2}{c|}{ } \\ \hline 
		Start & 
		 &  & 
		ResyncEvent &  & 
		ResyncEvent &  &
		S1 & &
		 &  &
		&  & 
		\multicolumn{2}{c|}{ Error 1} \\ \hline 
		S1 & 
		 &  & 
		 &  & 
		 &  &
		 & End 1 &
		End 1 or S11 &  &
		End 1 & S11 & 
		\multicolumn{2}{c|}{ } \\ \hline 
		S11 & 
		 &  & 
		 &  & 
		\textit{ResyncEvent} & \textit{Start} &
		 & &
		 &  &
		 &  & 
		\multicolumn{2}{c|}{ } \\ \hline 
		
		ResyncEvent & 
		
		\multicolumn{14}{c|}{ Start } \\ \hline 
		
		End1 & 
		\multicolumn{14}{c|}{ Start } \\ \hline 		
		Error1 & 
		\multicolumn{14}{c|}{  } \\ \hline 
		%\caption[State transition table] {Memory controll unit state transition table}
	\end{longtable}
	\renewcommand{\arraystretch}{1.0}
	* Signal appears for state \textit{S11} alone. 
%	\renewcommand{\textwidth}{\temp}
\end{landscape}

\subsection{Timestamp Generator}

\subsection{AMBA}
