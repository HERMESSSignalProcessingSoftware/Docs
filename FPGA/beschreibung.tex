\section{Component: Stamp}
\label{bes:Stamp}
The \textit{component STAMP} describes the logic used for the communication with the ADS 1147. (Look at figure \ref{fig:stamp}).  
SPI is used for the communication with the ADS 1147 analog - digial converter (ADC). The configuration of the ADC will be done from the microcontroller subsystem (MSS), therefore it will write to the \textit{STAMP} via APB3. \\ \\
During the measurment operations, this component will trigger an synchron signal \textit{New Avail}, signaling that the values of the ADCs are loaded successfully into the \textit{STAMP} component. On a rising edge of \textit{new avail} the data will be shown in the 64 bit databus. \\\\
The strain gauge rosetts (SGR) will be sampled with 2000 SP / s. The temperature sensor will be sampled with 10 SP/s. 
% TODO: Mehr dazu schreiben, Jonathans komponente
\section{Komponente: Memory}
\label{bes:memory}
\subsection{Controll Unit}
The \textit{Controll Unit} (CU) implements the syncrhonizer finit statemachine to resolve smaller problems during measurment operations. The CU will be initialized with a falling edge on $\overline{reset}$. After that it is able to do parallel APB3 and measurment to memory operations. \\ \\
The data will be stored on a 1Gbit external flash memory, connected to the MSS. 
\begin{figure}[htb] 
	\begin{center}
		\begin{tikzpicture}[node distance = 4cm, black, ultra thick]
			\node[state] (start) at (0,0) {Start};
			\node[state, above right of = start, node distance = 8cm] (ResyncEvent) {Resync Event};
			\node[state, below of = ResyncEvent, node distance = 8cm] (S11) {S11};
			\node[state, below left of = S11] (S1) {S1};
			\node[red!50, state, above left of = start] (Error1) {Error 1};
			\node[state, below of = start, node distance = 8cm] (End1) {End 1};
			% Draw lines 
			\coordinate (tempStart) at ($(start) - (2,-1)$);
			\path[->, draw, black] (tempStart) -- node[sloped, anchor=center, above, align=left]{$\overline{reset}$} (start);
			\path[->] (start) edge[loop left]  node[sloped, anchor=center, above,text width=3.5cm, align=left] {all other}(start);
			\path[->] (start) edge[bend left] node[sloped, anchor=center, above,text width=3.5cm, align=left]{\small{\# Request Resync = $X_1$\\ AND \\ ResyncTimer = 0}} (ResyncEvent);
			\path[->] (ResyncEvent) edge[bend left] (start);
			\path[->] (start) edge[bend left] node[sloped, anchor=center, above,text width=3.2cm, align=left] {\small{\# new Avail $>$ 0}} (S1);
			\path[->] (S1) edge[loop left] node[sloped, anchor=center, above, align = left] {\small{all other}} (S1);
			\path[->] (S1) edge[bend left] node[sloped, anchor=center, below, text width=3.2cm,align = left] {\small{\# New Avail = 6 \\ OR \\ $~$\\ \# New Avail $\ge~X_2$ \\ AND \\ WaitingTimer = 0 }} (End1);
			\path[->] (S1) edge[bend right] node[sloped, anchor=center, below, text width=3.2cm,align = left] {\small{WaitingTimer = 0 \\ AND \\ \# New Avail < $X_2$}}(S11);
			\path[->] (S11) edge[bend right] node[sloped, anchor=center, below, text width=3.2cm,align = left] {ResyncTimer = 0} (ResyncEvent);
			\path[->] (S11) edge[bend right] node[sloped, anchor=left, above, text width=3.2cm,align = left] {\small{ResyncTimer $\neq$ 0}}(start);
			\path[->] (start) edge[bend right] node[sloped, anchor=center, above, text width=3.2cm,align = left]{\small{SyncTimer = 0}} (Error1);
			\path[->] (End1) edge[bend left] (start);
		\end{tikzpicture}
	\end{center}
	\label{fig:MemCu}
	\caption[Memory controll unit statemachine]{Memory controll unit finit statemachine}
\end{figure}
\begin{landscape}
	\renewcommand{\arraystretch}{1.5}
	\setlength\LTcapwidth{\textwidth} % default: 4in (rather less than \textwidth...)
	\setlength\LTleft{0pt}            % default: \parindent
	\setlength\LTright{0pt}           % default: \fill
	\begin{longtable}[ht]{@{\extracolsep{0pt}}|l||*{14}{l|}}\hline 
		\textbf{Status} & 
		\multicolumn{2}{c|}{\textbf{$\overline{reset}$}} & 
		\multicolumn{4}{c|}{\textbf{\# Request Resync} \&\& \textbf{ ResyncTimer} } & \multicolumn{2}{c|}{\textbf{\# New Avail}} & 
		\multicolumn{4}{c|}{\textbf{WaitingTimer} \& \& \textbf{\# New Avail}} & 
		\multicolumn{2}{c|}{\textbf{SyncTimer} } \\ \hline 
		&
		\multicolumn{2}{c|}{} & 
		\multicolumn{2}{c|}{\textbf{\# Request Resync}} & \multicolumn{2}{c|}{\textbf{ ResyncTimer*} } & \multicolumn{2}{c|}{} & 
		\multicolumn{2}{c|}{\textbf{WaitingTimer}} & \multicolumn{2}{c|}{\textbf{\# New Avail}} & 
		\multicolumn{2}{c|}{} \\ \hline 
		\textbf{Wert} & 
		0 & 1 & 
		== $X_1$ & $\neq X_1$ & 
		== 0 & $\neq 0$ &
		> 0 & == 6 &
		== 0 & $\neq 0$ &
		> 0 \&\&  $\geq ~X_2$ & > 0 \&\& < $X_2$ & 
		\multicolumn{2}{c|}{ 0} \\ \hline 
		\endhead
		Undefined   & 
		Start &  &  & & &  & &  & &  & & & 
		\multicolumn{2}{c|}{ } \\ \hline 
		Start & 
		 &  & 
		ResyncEvent &  & 
		ResyncEvent &  &
		S1 & &
		 &  &
		&  & 
		\multicolumn{2}{c|}{ Error 1} \\ \hline 
		S1 & 
		 &  & 
		 &  & 
		 &  &
		 & End 1 &
		End 1 or S11 &  &
		End 1 & S11 & 
		\multicolumn{2}{c|}{ } \\ \hline 
		S11 & 
		 &  & 
		 &  & 
		\textit{ResyncEvent} & \textit{Start} &
		 & &
		 &  &
		 &  & 
		\multicolumn{2}{c|}{ } \\ \hline 
		
		ResyncEvent & 
		
		\multicolumn{14}{c|}{ Start } \\ \hline 
		
		End1 & 
		\multicolumn{14}{c|}{ Start } \\ \hline 		
		Error1 & 
		\multicolumn{14}{c|}{  } \\ \hline 
		%\caption[State transition table] {Memory controll unit state transition table}
	\end{longtable}
	\renewcommand{\arraystretch}{1.0}
	* Signal appears for state \textit{S11} alone. 
%	\renewcommand{\textwidth}{\temp}
\end{landscape}

\subsection{Timestamp Generator}
This component generates a timestamp, starting at \textit{enable} goes high. The internal prescaler is set to 24 which leads to a resolution of 500 ns.
\begin{equation}
	\begin{split}
		f_{cpu} = 50~MHz \\ 
		T_{cpu} = \frac{1}{f_{cpu}} = \frac{1}{50~MHz} = 20~ns \\
		T_{res} = \frac{1}{f_{cpu}} \cdot \left(24 + 1\right) = T_{cpu} \cdot 25\\
		T_{res} = 20~ns \cdot 25 = 500 ~ ns
	\end{split}
\label{eq:prescaler}
\end{equation}
The timestamp will be copied in the 32 bit timestamp signal with the next rising edge of \textit{clk}. This leads to the decrement of the prescaler by one. This behaviour is mentiond in equation \ref{eq:prescaler} above.  \\ \\
Timestamp overflow is not covered by the system. It will reach this condition after 71.58 minutes.
\begin{equation}
	\begin{split}
		T_{overflow} =  \frac{\frac{2^{32} - 1 ~ \mu s}{2}}{10^6 \frac{\mu s}{s}} = 2147~s\\
		T_{overflow, min} = \frac{T_{overflow}}{60 \frac{s}{min}} = \frac{2147~s}{60 \frac{s}{min}} = 35.79 ~min
	\end{split}
\end{equation}
In expectation of the flight time, there is more than enough space to prevent the system from reaching the overflow condition. 
\subsection{AMBA}
\begin{figure}[htb] 
	\begin{center}
		\begin{tikzpicture}[node distance = 4cm, black, ultra thick]
			\node[state] (apbwaiting) {ABPWaiting};
			\node[state, below left of = apbwaiting, node distance = 5cm] (apbwriting) {APBWriting};
			\node[state, below right  of = apbwaiting, node distance = 5cm] (apbreading) {APBReading};
			\path[->] (apbwaiting) edge[bend right] node[sloped, anchor=center, above, text width=3.6cm,align = left] {PENABLE = high \\ \&\& PWRITE = high \\ \&\& PSEL = high} (apbwriting);
			\path[->] (apbwaiting) edge[bend left] node[sloped, anchor=center, above, text width=3.2cm,align = left] {PWRITE = low \\ \&\& PSEL = high} (apbreading);
			%
			\path[->] (apbwriting) edge[bend right] (apbwaiting);
			\path[->] (apbreading) edge[bend left] (apbwaiting);
		\end{tikzpicture}
	\end{center}
	\caption[AMBA finit statemachine]{AMBA finit statemachine}
	\label{fig:memamba}
\end{figure} 
\noindent
The AMBA connection implements a slightly smaller finit state machine, which is shown in figure \ref{fig:memamba}. 
Everytime the state machine reaches a state, the state will be excecuted after that the FSM will return to the waiting state. During the exceution the address will be used to access the corresponding register, see table \ref{tbl:registeraddress}.
