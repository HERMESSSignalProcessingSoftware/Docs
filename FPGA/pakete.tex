\usepackage{tikz}
\usepackage{xcolor} % Farben
\usepackage{color, colortbl}
\usetikzlibrary{decorations.pathmorphing}
\usetikzlibrary{decorations.pathreplacing}
\usetikzlibrary{arrows, decorations.markings}
\usetikzlibrary{positioning, fit, shapes.geometric, calc}
\usepackage{pgfplots}
\usepackage{pstricks-add}
\usepackage{amsmath}
\usepackage{amssymb}
\usepackage{float}
\usepackage{caption}
\usepackage{ragged2e}
\usepackage{stix}
\usepackage{booktabs}
\usepackage{multicol}
\usepackage{tabularx}
\usepackage[onehalfspacing]{setspace}
\usepackage[sfdefault]{noto} % Standardschrift aendern
\usepackage[utf8]{inputenc}
\usepackage[ngerman]{babel} % Woerterbuch
\definecolor{tableHeader}{RGB}{255,238,205}
\definecolor{white}{RGB}{255,255,255}
\usepackage[left=2.50cm, right=2.50cm, top=2.50cm, bottom=3.0cm]{geometry} % Seitengeometrie
\newcommand*\circled[1]{\tikz[baseline=(char.base)]{
		\node[shape=circle,draw,inner sep=1pt] (char) {#1};}}
\newcommand{\mline}[1]{\begin{tabular}{@{}l@{}}#1\end{tabular}}
\usepackage[
backend=biber,
style=authoryear,
sortlocale=de_DE,
natbib=true,
url=false, 
doi=true,
eprint=false,
%backref=true %% In den Literaturangaben anzeigen, an welchen Stellen/Seiten das Zitat gesetzt ist
]{biblatex}
\addbibresource{documentation.bib} 
\usepackage{pdflscape}
\newcommand{\quotes}[1]{\glqq#1\grqq}
\usepackage{longtable}
\usepackage{multirow}
%\usetikzlibrary{shapes,arrows}
%\usepackage{amsmath,bm,times}
%\usepackage{verbatim}