\usepackage{tikz}
\usepackage{xcolor} % Farben
\usepackage{color, colortbl}
\usetikzlibrary{decorations.pathmorphing}
\usetikzlibrary{decorations.pathreplacing}
\usetikzlibrary{arrows, decorations.markings}
\usetikzlibrary{positioning, fit, shapes.geometric, calc}
\usepackage{pgfplots}
\usepackage{pstricks-add}
\usepackage{amsmath}
\usepackage{amssymb}
\usepackage{float}
\usepackage{caption}
\usepackage{ragged2e}
\usepackage{stix}
\usepackage{booktabs}
\usepackage{multicol}
\usepackage{tabularx}
\usepackage[onehalfspacing]{setspace}
\usepackage[sfdefault]{noto} % Standardschrift aendern
\usepackage[utf8]{inputenc}
\usepackage[english]{babel} % Woerterbuch
\definecolor{tableHeader}{RGB}{255,238,205}
\definecolor{white}{RGB}{255,255,255}
\usepackage[left=2.50cm, right=2.50cm, top=2.50cm, bottom=3.0cm]{geometry} % Seitengeometrie
\newcommand*\circled[1]{\tikz[baseline=(char.base)]{
		\node[shape=circle,draw,inner sep=1pt] (char) {#1};}}
%\newcommand{\mline}[1]{\begin{tabular}{@{}l@{}}#1\end{tabular}}
%\usepackage[
%backend=biber,
%style=authoryear,
%sortlocale=de_DE,
%natbib=true,
%url=false, 
%doi=true,
%eprint=false,
%backref=true %% In den Literaturangaben anzeigen, an welchen Stellen/Seiten das Zitat gesetzt ist
%]{biblatex}
%\addbibresource{documentation.bib} 
\usepackage{pdflscape}
\newcommand{\quotes}[1]{\glqq#1\grqq}
\usepackage{longtable,lscape}
\usepackage{multirow}
%\usetikzlibrary{shapes,arrows}
%\usepackage{amsmath,bm,times}
%\usepackage{verbatim}
\newcommand{\bitrect}[2]{
	\begin{pgfonlayer}{foreground}
		\draw [thick] (0,0) rectangle (#1,1);
		\pgfmathsetmacro\result{#1-1}
		\foreach \x in {1,...,\result}
		\draw [thick] (\x,1) -- (\x, 0.8);
	\end{pgfonlayer}
	%  \node [below left, align=right] at (0,0) {Type \\ Reset};
	\bitlabels{#1}{#2}
}
\newcommand{\rwbits}[3]{
	\draw [thick] (#1,0) rectangle ++(#2,1) node[pos=0.5]{#3};
	\pgfmathsetmacro\start{#1+0.5}
	\pgfmathsetmacro\finish{#1+#2-0.5}
	%  \foreach \x in {\start,...,\finish}
	%    \node [below, align=center] at (\x, 0) {R/W \\ 0};
}
\newcommand{\robits}[3]{
	\begin{pgfonlayer}{background}
		\draw [thick, fill=lightgray] (#1,0) rectangle ++(#2,1) node[pos=0.5]{#3};
	\end{pgfonlayer}
	\pgfmathsetmacro\start{#1+0.5}
	\pgfmathsetmacro\finish{#1+#2-0.5}
	%  \foreach \x in {\start,...,\finish}
	%    \node [below, align=center] at (\x, 0) {RO \\ 0};
}
\newcommand{\bitlabels}[2]{
	\foreach \bit in {1,...,#1}{
		\pgfmathsetmacro\result{#2}
		\node [above] at (\bit-0.5, 1) {\pgfmathprintnumber{\result}};
	}
}
\usepackage{bytefield}
\newcommand{\memsection}[4]{
	\bytefieldsetup{bitheight=#3\baselineskip}	% define the height of the memsection
	\bitbox[]{8}{
		\texttt{0x\uppercase{#1}}	 % print end address
		\\ \vspace{#3\baselineskip} \vspace{-2\baselineskip} \vspace{-#3pt} % do some spacing
		\texttt{0x\uppercase{#2}} % print start address
	}
	\bitbox{16}{#4} % print box with caption
}