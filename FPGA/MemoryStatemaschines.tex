% Memory Statemachines 
%type typeCUState is (Waiting, Preparing, CopyShadow, Init, Init1, STAMP1, STAMP2, STAMP3, STAMP4, STAMP5, STAMP6, END1, END2, END3, Locked);
%    signal ControllUnitState        : typeCUState;
%    type typeStampSubstate is (Prepare, Working, Working1);
%    signal ControllUnitSubState     : typeStampSubstate;
\section{Memory Statemachines}
\begin{landscape}
\begin{figure}[htb] 
\begin{center}
\begin{tikzpicture}[node distance = 4cm, black, ultra thick]

\node[state] (Waiting) at (0, 1) {Waiting};
\node[state, above right of = Waiting] (Preparing) {Preparing};
\node[state, above right of = Preparing] (Init) {Init};
\node[state, right of = Init] (Init1) {Init1};
\node[state, right of = Init1] (CopyShadow) {CopyShadow};
\node[state, below right of = CopyShadow] (STAMP1) {STAMP1};
\node[state, below right of = STAMP1] (STAMP2) {STAMP2};
\node[state, below of = STAMP2] (STAMP3) {STAMP3};
\node[state, below left of = STAMP3] (STAMP4) {STAMP4};
\node[state, below left of = STAMP4] (STAMP5) {STAMP5};
\node[state, left of = STAMP5] (STAMP6) {STAMP6};
\node[state, left of = STAMP6] (END1) {END1};
\node[state, above left of = END1] (END2) {END2};
\node[state, above left of = END2] (END3) {END3};
\node[state, node distance = 5cm] (Locked) at ($(END3) !.5! (Waiting) + (3cm, 0cm)$) {Locked};

\coordinate (Start) at ($(Waiting) + (-2cm,0cm)$);
\coordinate (EnterLocked) at ($(Locked) + (3.5cm, 2cm)$);

%Start of FSM
\path[black, draw, ->]  (Start) -- node[sloped, anchor=center, above] {$\overline{reset}$}  (Waiting);

%Enter Locked
\path[black, draw, ->]  (EnterLocked) -- node[sloped, anchor=center, above] {CSR(2) = High}  (Locked);
%Stay at locked
\path[->] (Locked) edge[loop above] node[sloped, anchor=center, above] {CSR(2) = High} (Locked); 
%Leave locked
\path[->] (Locked) edge[bend left] node[sloped, anchor=center, below] {CSR(2) = Low} (Waiting);

%Normal FSM action
\path[->] (Waiting) edge[bend left] node[sloped, anchor=center, above, align=left] {$\text{drdy}_{1-6}$ = High \\WdtGo = High} (Preparing);
\path[->] (Preparing) edge[bend left] node[sloped, anchor=center, above, align=left] {} (Init);
\path[->] (Init) edge[bend left] node[sloped, anchor=center, above, align=left] {} (Init1);
\path[->] (Init1) edge[bend left] node[sloped, anchor=center, above, align=left] {} (CopyShadow);
\path[->] (CopyShadow) edge[bend left] node[sloped, anchor=center, above, align=left] {} (STAMP1);
\path[->] (STAMP1) edge[bend left] node[sloped, anchor=center, above, align=left] {} (STAMP2);
\path[->] (STAMP2) edge[bend left] node[sloped, anchor=center, above, align=left] {} (STAMP3);
\path[->] (STAMP3) edge[bend left] node[sloped, anchor=center, above, align=left] {} (STAMP4);
\path[->] (STAMP4) edge[bend left] node[sloped, anchor=center, above, align=left] {} (STAMP5);
\path[->] (STAMP5) edge[bend left] node[sloped, anchor=center, above, align=left] {} (STAMP6);
\path[->] (STAMP6) edge[bend left] node[sloped, anchor=center, above, align=left] {} (END1);
\path[->] (END1) edge[bend left] node[sloped, anchor=center, below, align=left] {MemoryCnt = 128} (END2);
\path[->] (END2) edge[bend left] node[sloped, anchor=center, below, align=left] {TempCounter = 129} (END3);

%Path of END1 to CopyShadow
\path[->] (END1) edge[bend right] node[sloped, anchor=center, above, align=left] {MemoryCnt $\neq$ 128 AND \\$(\text{drdy}_{1-6}$ = High  OR \\ WdtGo = High)} (CopyShadow);
%Stay at END2 
\path[->] (END2) edge[loop above] node[sloped, anchor=center, above] {TempCounter $\neq$ 129} (END2); 
% Normal movement form END3 to Waiting
\path[->] (END3) edge[bend left] node[sloped, anchor=center, below, align=left] {} (Waiting);

\end{tikzpicture}
\caption{Controllunit statemachine} 
\end{center}		  
\end{figure}

\renewcommand{\arraystretch}{1.5} 
\begin{longtable} [htb] { | p{3.5cm} | *{11}{p{1.5cm} |} } \hline
		\multirow{2}{*}{\textbf{Ausgangszustand}}  & 
		\multicolumn{2}{p{3cm} |}{  \textbf{$\text{Reset}_n$}} & 
		\multicolumn{2}{p{3cm} |}{  \textbf{$\text{drdy}_{1-6}$}} & 
		\multicolumn{2}{p{3cm} |}{  \textbf{MemoryCnt}} &
		\multicolumn{2}{p{3cm} |}{  \textbf{TempCount}} &
		\multicolumn{2}{p{3cm} |}{  \textbf{CSR(2)}} &
		\textbf{Default}\\ \cline{2-12}
\endhead
 			& 0 	& 1 & 0 & 1 & \multicolumn{2}{c|}{--} & \multicolumn{2}{c|}{--} & 0 & 1 & -- \\ \hline
Nicht definiert 	& 	& \small{Waiting} &  & & \multicolumn{2}{c|}{} & \multicolumn{2}{c|}{} & & Locked &  \\ \hline
Locked 		& 	& &  & & \multicolumn{2}{c|}{} & \multicolumn{2}{c|}{} & Waiting & Locked &  \\ \hline
Waiting 		& 	& & & \small{Preparing} & \multicolumn{2}{c|}{} & \multicolumn{2}{c|}{} & & Locked &  \\ \hline
Preparing		& 	& & & & \multicolumn{2}{c|}{} & \multicolumn{2}{c|}{} & & Locked & \small{Init} \\ \hline
Init			& 	& & & & \multicolumn{2}{c|}{} & \multicolumn{2}{c|}{} & & Locked & \small{Init1} \\ \hline
CopyShadow	& 	& & & & \multicolumn{2}{c|}{} & \multicolumn{2}{c|}{} & & Locked & \small{Stamp1} \\ \hline
Stamp1		& 	& & & & \multicolumn{2}{c|}{} & \multicolumn{2}{c|}{} & & Locked & \small{Stamp2} \\ \hline
Stamp2		& 	& & & & \multicolumn{2}{c|}{} & \multicolumn{2}{c|}{} & & Locked & \small{Stamp3} \\ \hline
Stamp3		& 	& & & & \multicolumn{2}{c|}{} & \multicolumn{2}{c|}{} & & Locked & \small{Stamp4} \\ \hline
Stamp5		& 	& & & & \multicolumn{2}{c|}{} & \multicolumn{2}{c|}{} & & Locked & \small{Stamp6} \\ \hline
Stamp6		& 	& & & & \multicolumn{2}{c|}{} & \multicolumn{2}{c|}{} & & Locked & \small{End1} \\ \hline
End1			& 	& & & \tiny{CopyShadow} & \multicolumn{2}{c|}{END2 bei 128} & \multicolumn{2}{c|}{} & & Locked & \\ \hline
End2			& 	& & &  & \multicolumn{2}{c|}{} & \multicolumn{2}{c|}{END3 bei 129} & & Locked & \small{End2}\\ \hline
End3			& 	& & &  & \multicolumn{2}{c|}{} & \multicolumn{2}{c|}{} & & Locked & \small{Waiting}\\ \hline
\caption{Memory Controllunit Statemachine Übergangstabelle}
\end{longtable}
\renewcommand{\arraystretch}{1.0} 
\end{landscape}
\section{Memory Controllunit FSM description}
\subsection{Locked} 
If controllunit status register (\textit{CSR}) bit 2 is set, the controllunit will enter a \textit{Locked} state. It leaves this state of in activity with CSR bit 2 transmission to a low state. The next state is \textit{Waiting}. \\\\
Note: Setting CSR(2) to high will cancel all operations of the controllunit except of the AMBA operations which are not part of this statemachine. 
\subsection{Waiting} 
To begin a new complete page (512 byte of memory) the controlunit will wait for \textit{Dataready} (\textit{drdy}) 1 to 6 to switch to a high state. If all of them are at a high state or the component \textit{Watchdog} will present its \textit{WatchdogGo} signal on a high state the waiting state will be left. \\\\
Befor lefting this waiting state the machine will be calculating the new page address of the external memory. The address of the internal memory will be calculated during this phase. 
\subsection{Preparing} 
\textit{CSR}(30) will be set to 0. TODO: ADD Name of this bit here! \\\\
To fill the two external memories evenly, the page address will be increased at every second time we reached this state. To prevent this machine form an invalid memory access the address will be set to zero. 
\subsection{Init}
Signal \textit{WriteEnable} transmission to high, enables the internal RAM.\\
Signal \textit{InternalAddr2Mem} will be set to 0x01. \\
Signal \textit{InternalData2Mem} will be setup with SPI command (0x12) at bits 24 to 31. The lower bits will be setup to the page address, saved in register \textit{StampFSMR1} later used as \textit{R1}. \\
Register \textit{StampFSMPC}, later referred as \textit{PC} set to the address of the SPI command (0x00000001). 
\subsection{Init1} 
Signal \textit{InternalAddr2Mem} will be set to 0x02. 
Signal \textit{InternalData2Mem} bits 24 to 31 holdes the last byte (bit 0 to 7) of \textit{R1} followed by an hard coded value of 0x7FFFFF to mark the page as used on the external memory. 
\subsection{CopyShadow} 
Signal \textit{datareadyreset} set to high level.\\\\
Copying the data of each stamp to the corresponding STAMP shadow registers. \\
All six databus will be mapped to 384 bit wide bus at the component \textit{Memory}. \\\\
\begin{longtable} [htb] { | p{0.45 \linewidth} | p{0.45\linewidth} | } \hline
		\textbf{Signal from} & \textbf{Connected to} \\ \hline
\endhead
	STAMP 1 & Databus 320 to 383 \\ \hline 
	STAMP 2 & Databus 256 to 319 \\ \hline 
	STAMP 3 & Databus 192 to 255 \\ \hline
	STMAP 4 & Databus 128 to 191 \\ \hline
	STAMP 5 & Databus 64 to 127 \\ \hline
	STAMP 6 & Databus 0 to 63 \\ \hline 	
\end{longtable} 
Register \textit{StampXShadowReg1} holds the bits 32 to 63 of the incoming 64 bit bus. \\ 
Register \textit{StampXShadowReg2} holds the bits 0 to 31 of the incoming 64 bit bus. 
Note: $\{ \text{X} : \textit{X} \in  \{1, ..., 6\} \}$ 
\subsubsection{STAMP 1}
\begin{figure}[htb] 
\begin{center}
\begin{tikzpicture}[node distance = 4cm, black, ultra thick]

\node[state] (Prepare) at (0,0) {Prepare};
\node[state, right of = Prepare] (Working) {Working};
\node[state, right of = Working] (Working1) {Working1};
\node[state, below of = Working, align = left] (STAMP2prep) {Stamp 2 \\ Prepare};
\coordinate (start) at ($(Prepare) + (-1.8cm, 0cm) $);
\path[black, ultra thick, draw, ->] (start) -- (Prepare);
\path[->] (Prepare) edge[bend left] node[sloped, anchor=center, below, align=left] {} (Working);
\path[->] (Working) edge[bend left] node[sloped, anchor=center, below, align=left] {} (Working1);
\path[->] (Working1) edge[bend left] node[sloped, anchor=center, below, align=left] {} (STAMP2prep);
\end{tikzpicture}
\end{center}
\caption{Stamp 1 substates}
\end{figure}
The transmission of \textit{Prepare} to \textit{Working1} will be after each clock cycle. \\\\ 
In state \textit{Prepare}  the system will enable the writing to the internal RAM by setting signal \textit{WriteEnable} to high. 
Also signal \textit{getTime} is set to a high level, to gather the current timestamp form the timestamp module. \\\\
In state \textit{Working} the system is transmitting the Stamp1ShadowReg1 to the internal RAM. 
In state \textit{Working1} the system is transmitting the Stamp1ShadowReg2 to the internal RAM. After state \textit{Working1} this system will be switch to state \textit{Stamp2} with substate \textit{Prepare}. 
\subsection{STAMP 2}
\begin{figure}[htb] 
\begin{center}
\begin{tikzpicture}[node distance = 4cm, black, ultra thick]

\node[state] (Prepare) at (0,0) {Prepare};
\node[state, right of = Prepare] (Working) {Working};
\node[state, right of = Working] (Working1) {Working1};
\node[state, below of = Working, align = left] (STAMP3prep) {Stamp 3 to 6 \\ Working};
\coordinate (start) at ($(Prepare) + (-1.8cm, 0cm) $);
\path[black, ultra thick, draw, ->] (start) -- (Prepare);
\path[->] (Prepare) edge[bend left] node[sloped, anchor=center, below, align=left] {} (Working);
\path[->] (Working) edge[bend left] node[sloped, anchor=center, below, align=left] {} (Working1);
\path[->] (Working1) edge[bend left] node[sloped, anchor=center, below, align=left] {} (STAMP3prep);
\end{tikzpicture}
\end{center}
\caption{Stamp 2 substates}
\end{figure}
In state \textit{Prepare} the system switches signal \textit{getTime} to a low level. \\\\
In state \textit{Working} the system is transmitting the Stamp1ShadowReg1 to the internal RAM. \\\\
In state \textit{Working1} the system is transmitting the Stamp1ShadowReg2 to the internal RAM. After state \textit{Working1} this system will be switch to state \textit{Stamp3} with substate \textit{Working}. 

\subsection{STAMP 3 to 6} 
\begin{figure}[htb] 
\begin{center}
\begin{tikzpicture}[node distance = 4cm, black, ultra thick]

%\node[state] (Prepare) at (0,0) {Prepare};
\node[state, right of = Prepare] (Working) at (0,0) {Working};
\node[state, right of = Working] (Working1) {Working1};
\node[state, right of = Working1, align = left] (STAMP2prep) {Stamp X +1 \\ Working\\ End1};
\coordinate (start) at ($(Working) + (-1.8cm, 0cm) $);
\path[black, ultra thick, draw, ->] (start) -- (Working);
%\path[->] (Prepare) edge[bend left] node[sloped, anchor=center, below, align=left] {} (Working);
\path[->] (Working) edge[bend left] node[sloped, anchor=center, below, align=left] {} (Working1);
\path[->] (Working1) edge[bend left] node[sloped, anchor=center, below, align=left] {} (STAMP2prep);
\end{tikzpicture}
\end{center}
\caption{Stamp 3 to 6 substates}
\end{figure}
In state \textit{Working} the system is transmitting the StampXShadowReg1 to the internal RAM. 
In state \textit{Working1} the system is transmitting the StampXShadowReg2 to the internal RAM. After state \textit{Working1} this system will be switch to state \textit{StampX + 1} with substate \textit{Working}. \\\\
Note: $\{\text{X} : \text{X} \in \{3, 4, 5, 6\} \} $\\
Note: If X = 6 the system will change the state to \textit{End1} substate \textit{Working}.

\subsection{End1}
\begin{figure}[htb] 
\begin{center}
\begin{tikzpicture}[node distance = 4cm, black, ultra thick]

%\node[state] (Prepare) at (0,0) {Prepare};
\node[state, right of = Prepare] (Working) at (0,0) {Working};
\node[state, right of = Working] (Working1) {Working1};
\node[state, right of = Working1, align = left] (STAMP2prep) {End2 \\ Prepare};
\coordinate (start) at ($(Working) + (-1.8cm, 0cm) $);
\path[black, ultra thick, draw, ->] (start) -- (Working);
%\path[->] (Prepare) edge[bend left] node[sloped, anchor=center, below, align=left] {} (Working);
\path[->] (Working) edge[bend left] node[sloped, anchor=center, below, align=left] {} (Working1);
\path[->] (Working1) edge[bend left] node[sloped, anchor=center, below, align=left] {} (STAMP2prep);
\end{tikzpicture}
\end{center}
\caption{End 1 substates}
\end{figure}
\noindent
In substate \textit{Working} the system writes 0x0 to the next address after the last value of state \textit{Stamp6} \textit{Working1} to mark the end of a dataset. \\
Substate \textit{Working1} the system saves the value of the current memory counter, referring to the last address the system used in internal RAM, to Register \textit{StampFSMR2}, later referred to as \textit{R2}. \\
Signal \textit{WriteEnable} set to low level. \\\\ 
If variable \textit{MemoryCnt} is 128, the system will be switch to state \textit{End2} substate \textit{Prepare}. In all other cases the system will be waiting like state \textit{Waiting} but it switches to state \textit{CopyShadow}. 
\subsection{End2} 
\begin{figure}[htb] 
\begin{center}
\begin{tikzpicture}[node distance = 4.5cm, black, ultra thick]

\node[state] (Prepare) at (0,0) {Prepare};
\node[state, right of = Prepare] (Working)  {Working};
\node[state, right of = Working] (Working1) {Working1};
\node[state, above of = Working1, align = left] (E3P) {End3\\ Prepare};
\coordinate (start) at ($(Prepare) + (-1.8cm, 0cm) $);
\path[black, ultra thick, draw, ->] (start) -- (Prepare);
\path[->] (Prepare) edge[bend left] node[sloped, anchor=center, below, align=left] {} (Working);
\path[->] (Working) edge[bend left] node[sloped, anchor=center, above, align=left] {TempCounter = 129} (E3P);
\path[->] (Working) edge[bend left] node[sloped, anchor=center, above, align=left] {TempCounter $\neq$ 129} (Working1);
\path[->] (Working1) edge[bend left] node[sloped, anchor=center, below, align=left] {InternalBusy = Low} (Prepare);
\path[->] (Working1) edge[loop above] node[sloped, anchor=center, above] {InternalBusy = High} (Working1); 
\end{tikzpicture}
\end{center}
\caption{End 2 substates}
\end{figure}
\noindent
In substate \textit{Prepare} the system sets the signal \textit{InternalAddr2Mem} to PC and markes SPI as busy by setting the bit 29 of CSR to high. \\\\
Substate \textit{Working} is used to determinate the next state. State \textit{End3} substate \textit{Prepare} will be next if variable \textit{TempCounter} is 129. In all other cases the system will set \textit{TempCounter} to the value of \textit{PC} and increase the value of \textit{TempCounter} by one. After increasing the value will be assigned to register \textit{PC}. \\\\
In substate \textit{Working1} the system sets signal \textit{enableSPI} to high level. If signal \textit{InternalBusy} is not set, the system
writes the content of Signal / Register \textit{InternalDataFromMemory} to the register \textit{SPITransmitReg}. This is triggering an SPI transmission. The substate will changed to \textit{Prepare} while \textit{InternalBusy} is not set. If signal \textit{InternalBusy} is set, the substate will reset the signal \textit{enableSPI} to low. 

\subsection{End3}
This state has not substates, it is used to clean up the state machine after transmitting its data to the memory controllers. \\
Signal \textit{enableSPI} is set to low level. \\
Variable \textit{TempCounter} is set to zero. \\
SPI busy marker at register \textit{CSR} bit 29 is set to zero.\\ 
The next SPI address is calculated. If \textit{SPIaddr} is zero, the next one is one. Otherwise, \textit{SPIAddr} is one the next one is zero. This will be saved at register \textit{CSR} bit 1 and 31 as \textit{current SPI address indicator}. 

\section{APB3 FSM}


\begin{figure}[htb] 
\begin{center}
\begin{tikzpicture}[node distance = 4.5cm, black, ultra thick]

\node[state] (AW) at (0,0) {APB3 Waiting};
\node[state, below right  of = AW] (AWrite)  {APB3 Writing};
\node[state, below left of = AW] (ARead) {APB3 Reading};
\path[->] (AW) edge[bend right] node[sloped, anchor=center, above, align=left] {PSEL = High \\ PWrite = Low } (ARead);
\path[->] (AW) edge[bend left] node[sloped, anchor=center, above, align=left] {PSEL = High \\ PWrite = High \\ PEnable = High } (AWrite);
\path[->] (AWrite) edge[bend left] node[sloped, anchor=center, above, align=left] { } (AW);
\path[->] (ARead) edge[bend right] node[sloped, anchor=center, above, align=left] { } (AW);

\path[->] (AW) edge[loop above] node[sloped, anchor=center, above] {} (AW); 
\end{tikzpicture}
\end{center}
\caption{APB3 States}
\end{figure}








